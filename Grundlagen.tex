\chapter{Theoretische Grundlagen}

\section{Stabilität in Inselnetzen}

\section{Speichertechnologien in Inselnetzen}

\paragraph{Wozu Speicher?}
Für diese Projektarbeit soll ein autarkes Inselnetz mit regenerativer Energieeerzeugung 
modelliert und simuliert werden.
Aus verschiedenen Gründen, welche im Folgenden genauer erläutert werden sollen, ist der Einsatz von
Speichertechnologien für die Umsetzung eines solchen Inselnetzes zwingend notwendig.

Grundlegend ist eine lückenlose Energieversorgung innerhalb eines Netzes nur möglich wenn nahezu gleich viel Energie in das Netz
eingespeist und abgenommen wird.
Entscheidende Parameter für die Regelung der Energieerzeugung sind dabei vor Allem die Netzfrequenz 
und -spannung.
Das deutsche Verbundnetz ist dafür in vier Regelzonen unterteilt, welche widerrum in verschiedene
Bilanzkreise unterteilt sind.
Innerhalb dieser Bilanzkreise wird anhand von Vorraussagen für den nächsten Tag versucht
eingespeiste und entnommene Leistung auszuregeln.
Durch den schwankenden Leistungsbedarf sind Abweichungen hier allerdings die Regel.
Bei der Betrachtung eines regenerativen Inselnetzes kommt die volatile Natur von regenerativen Energieerzeugern
als weiterer Faktor hinzu und erschwert eine korrekte Vorraussage enorm.
Diese Abweichungen der tatsächlich benötigten Leistung von der bereitgestellten führen zu Frequenzschwankungen
welche sich wiederrum negativ auf die Netzstabilität auswirken.
Zum Ausgleich dieser Schwankungen muss Regelernegie zur Verfügung gestellt werden.

Die benötigte Energie ist dabei unterteilt in Momentanreserve, Primärreserve, Sekundärreserve 
und Tertiärreserve.
Die Momentanreserve, welche geringe Frequenzabweichung direkt ausgleichen soll, wird im deutschen Verbundnetz
durch die Schwungmasse der Kraftwerks-Synchronmaschinen bereit gestellt.
Die Primärreserve hingegen greift erst ab einer Abweichung von $20 mHz$ und musss nach spätestens 30 Sekunden
sowie für mindestens 15 Minuten vollständig zur Verfügung stehen.
Hierfür werden heute schon zunehmend Batteriespeicher eingesetzt.
Zusätzlich wird nach 30 Sekunden die Sekundärregelleistung bereit gestellt, welche für eine Stunde verfügbar sein muss.
Nach 15 Minuten wird diese dann von der Tertiärregelreserve abgelöst, welche ebenfalls fürr eine Stunde verfügbar sein muss.
Die beiden letzten Regeleneergie-Kategorien werden in aller Regel von Kraftwerken in Teillast oder Kraftwerken mit kurzen Anfahrzeiten
erzeugt.
Zuletzt werden einzelne Netzabschnitte vom Netz getrennt um einen Zusammenbruch des Bilanzkreises zu vermeiden.
Dieses Vorgehen bleibt allerdings die äußerste Maßnahme und soll in aller Regel vermieden werden.

Für ein Inselnetz besteht nicht die Möglichkeit Teilnetze abzutrennen. 
Im schlimmsten Fall müssen einzelne Verbraucher und Erzeuger vom Netz getrennt werden um einen stabilen Betrieb zu sichern.
Um das weitestgehend zu vermeiden, ist eine Überdimensionierung von Erzeugern und Speichern meist das Mittel der Wahl.
Große Speicher zur Primärreserve bilden dabei einen wichtigen Grundpfeiler, wobei gerade Batteriespeicher 
auf Grund ihres schnellen Regelverhaltens in Frage kommen.

Zusätzlich sollen hier neben der Regelreserve der Vollständigkeit halber die Regelleistungsprodukte \texttt{Enhanced Frequency Response}
(EFR) und \texttt{Virtuelle Schwungmasse} (VSM) erwähnt werden.
Diese sind zwar noch nicht in den deutschen Markt integriert, könnten aber in der Zukunft eine wichtige Rolle spielen.

\paragraph{Rahmenbedingung und Bestimmungen}

\paragraph{Betriebsstrategien zu Batteriespeichern}