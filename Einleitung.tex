\chapter{Einleitung}

Im Zuge des jährlich ansteigenden Anteils an erneuerbaren Energien am deutschen Strommix gibt es auch große fortlaufende Änderungen im Bereich der dazugehörigen Speichertechnologien. 
Im vorliegenden Projekt geht es um die Be- und Entladestrategien bei Speicher in einem Inselnetz, das mit erneuerbaren Energien betrieben wird. 
Ziel ist es dabei ein simulationstüchtiges Modell in MatLab/Simulink zu erstellen. 
Das Modell soll dabei eine fiktive Kommune darstellen und insbesondere für den stabilen Netzbetrieb als auch die Betrachtung von Be- und Entladestrategien reale Einflussfaktoren simulierbar abbilden.
Eine grundsätzliche Problematik stellt dabei die Darstellung der verschiedenen Komponenten im Inselnetz dar. 
Hierzu zählen insbesondere die erneuerbaren Energieerzeuger, wie z.B. Windkraft, die Verbraucher, wie Haushalte, Gewerbe und Verkehr und die eingesetzten Speichersysteme.
Wichtig ist hierbei, die Komponenten entsprechend präzise abzubilden, ohne jedoch die Simulationstauglichkeit in MatLab/Simulink zu stören.
Relevante Fragestellungen sind hierbei die allgemeine Dimensionierung der Komponenten. 
Dazu gehört beispielsweise die Kapazität der Speicher, aber auch eine Ermittlung des Momentan- und Gesamtbedarfs der Stromverbraucher.
Außerdem umfasst dies die Auswahl von existierenden Speichertechnologien, da es Ziel ist, bereits existierende Technologien auf dem neusten Stand der Technik zu untersuchen.

