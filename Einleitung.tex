\chapter{Einleitung}

Im Zuge des jährlich ansteigenden Anteils an erneuerbaren Energien am deutschen Strommix gibt es auch große fortlaufende Änderungen im Bereich der dazugehörigen Speichertechnologien. 
Im vorliegenden Projekt geht es um die Be- und Entladestrategien bei Speicher in einem Inselnetz, das mit erneuerbaren Energien betrieben wird. 
Ziel ist es dabei ein simulationstüchtiges Modell in MatLab/Simulink zu erstellen. 
Das Modell soll dabei eine fiktive Kommune darstellen und insbesondere für den stabilen Netzbetrieb als auch die Betrachtung von Be- und Entladestrategien reale Einflussfaktoren simulierbar abbilden.
Das Vorgehen bei der Erarbeitung wird im Folgenden beschrieben.

Im ersten Teil geht es um die Beschreibung der theoretischen Grundlagen. 
Dabei werden verschiedene Aspekte genauer untersucht.
Die Stabilität in Inselnetzen ist dabei von grundlegender Bedeutung.
Darüber hinaus spielen die Verbraucher und die Speichertechnologien je eine genauer zu analysierende Rolle.

Des Weiteren wird der Stand der Technik beschrieben.
Hierbei wird zuerst dieser in Bezug auf Inselnetze allgemein genauer betrachtet.
Außerdem werden Speicher beschrieben, wobei drei Bereiche besonders relevant sind.
Dazu zählen sowohl Lithium-Ionen-Batterien als auch Redox-Flow-Batterien, die angeschaut und verglichen werden. 
Folglich stehen stationäre Batteriespeichersysteme im Fokus, da diese einen fundamentalen Baustein in Inselnetzen darstellen.

Im nächsten Abschnitt geht es um die Modellbeschreibung.
Diese umfasst Erzeuger, Verbraucher und Speicher, sowie die Netzmodellierung.
Dabei handelt es sich um ein bilanzielles und ein dreiphasiges Netzmodell.

Es folgt eine Auswertung von verschiedenen Simulationsdurchläufen, sowie ein Ausblick.
Zwei Aspekte stehen dabei im Vordergrund.
Einerseits sind mögliche Veränderungen und Verbesserungen der Modellierung von Interesse, andereseits sind Prognosen von technischen und allgemeinen Anforderungen an Inselnetzen in Zukunft zu beachten.
Es folgt ein Fazit.

\autoref{tab:Aufteilung} zeigt die Aufteilung der Kapitel innerhalb der Projektgruppe.

\begin{table}
	\begin{tabular}[htpb]{p{8cm}|p{3cm}}
		\textbf{Kapitel} & \textbf{Author} \\
		\hline
		Einleitung & Steffen Sterthoff \\
		Theoretische Grundlagen - Verbraucher in Inselnetzen & Steffen Sterthoff \\
		Theoretische Grundlagen - Stabilität in Inselnetzen & Magnus Müller \\
		Theoretische Grundlagen - Speichertechnologien in Inselnetzen & Darius Daub \\
		Stand der Technik & Steffen Sterthoff \\
		Modellbeschreibung - Erzeuger & Magnus Müller \\
		Modellbeschreibung - Verbraucher in Inselnetzen & Magnus Müller \\
		Modellbeschreibung - Speicher & Darius Daub \\
		Modellbeschreibung - Netzmodell - Bilanziell & Magnus Müller \\
		Modellbeschreibung - Netzmodell - Dreiphasig & Darius Daub \\
		Simulationsergebnisse - Bilanziell & Magnus Müller \\
		Simulationsergebnisse - Dreiphasig & Darius Daub \\
		Schlussteil & Steffen Sterthoff		
	\end{tabular}
 	\centering
	\caption{Aufteilung innerhalb der Projektgruppe}
	\label{tab:Aufteilung}
\end{table}
