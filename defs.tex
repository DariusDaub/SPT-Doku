%Versuch eine Grafik zu erstellen
\usepackage[dvipsnames,table]{xcolor}
\usepackage{siunitx}
\usepackage{pgf-spectra}
\usepackage{pgfplots}

\usepackage{graphicx}
\usepackage{subcaption}
%Zum Zentrieren von Tabellen-Captions
\usepackage[justification=centering]{caption}

% Beschriftungen für Tabellen kommen linksbündig über die Tabelle
\KOMAoption{captions}{tableheading,nooneline}
\setcaptionalignment[figure]{c}
\setcaptionalignment[table]{l}

%Für das einbinden von Code
\usepackage{listings}
\lstset{numbers=left, numberstyle=\tiny, numbersep=5pt}
\lstset{language=Perl}

% wird für die Titelseite benötigt
\usepackage{geometry}

% wird für die Nutzung der Hausschrift benötigt
\usepackage{fontspec}

% Standardpaket für Lokalisation, siehe Option "ngerman" oben
\usepackage{babel}
% Laden von optimierten Trennmustern
\babelprovide[hyphenrules=ngerman-x-latest]{ngerman}

% Standardpaket für mathematische Zusatzfunktionen; wenn Sie keine
% mathematischen Formeln brauchen, können Sie diese Zeile löschen
\usepackage{amsmath}
\usepackage{amssymb}
\usepackage{trfsigns}
\usepackage{mathtools}
% die Hauptschrift Libertinus
\usepackage{libertinus-otf}
% die "Schreibmaschinenschrift" Anonymous Pro, angepasst
\usepackage{AnonymousPro}
\setmonofont{AnonymousPro}[Scale=MatchLowercase,FakeStretch=0.85]

% etwas größerer Zeilenabstand als im Buchsatz
\linespread{1.1}

% Paket für Feinkorrekturen an der Typographie, das für ein ausgewogeneres
% Schriftbild sorgt
\usepackage{microtype}

% Paket für kontextsensitive Anführungszeichen
\usepackage{csquotes}
% Shortcut, damit aus dem eigentlich falschen Zeichen " richtige
% Anführungszeichen je nach Sprache werden
\MakeOuterQuote{"}

% Paket, das den Befehl \includegraphics ermöglicht
\usepackage{graphicx}

% komfortablere Aufzählungen als in Standard-LaTeX; ein Beispiel findet man in
% chap3.tex
\usepackage{enumitem}

% Paket für mehr als die üblichen Standardfarben
\usepackage[dvipsnames]{xcolor}
% Definition der "Hausfarben" der HAW
\definecolor{haw}{HTML}{003CA0}
\definecolor{haw2}{HTML}{0096D2}
\definecolor{haw3}{HTML}{A0BEDC}

% typographisch anspruchsvolle Tabellen; siehe chap3.tex
\usepackage{booktabs}

% zum Erstellen des Literaturverzeichnisses; der gängige Stil APA ist hier
% bereits eingestellt
\usepackage[style=ieee]{biblatex}
% eine Beispieldatei für ein Literaturverzeichnis
\addbibresource{bib.bib}

% für die Erzeugung der Grafiken in chap3.tex; wenn Sie PGF/TikZ nicht
% verwenden wollen, können Sie diese Zeilen entfernen
\usepackage{tikz}
% Zusatzbibliotheken für TikZ, die in den genannten Beispielen verwendet
% werden
\usetikzlibrary{calc,intersections,angles,3d}

% für die Erzeugung des Codeblocks in chap3.tex; wenn in Ihrer Arbeit keine
% Codeblöcke vorkommen, können Sie diese Zeilen entfernen
\usepackage{listings}
% Anpassung des Erscheinungsbildes des Codeblocks; mehr dazu in der
% Dokumentation des Pakets "listings"
\lstdefinestyle{mystyle}{
    backgroundcolor=\color{gray!20},
    keywordstyle=\color{haw2},
    numberstyle=\footnotesize\color{haw},
    basicstyle=\ttfamily\small,
    captionpos=t,
    frame=single,
    framerule=0pt,
    keepspaces=true,
    numbers=left,
    numbersep=6pt,
    belowcaptionskip=1em,
    aboveskip=\bigskipamount,
}
\lstset{style=mystyle}
% damit es "Codeblock" und nicht "Listing" heißt
\renewcommand{\lstlistingname}{Codeblock}

% für die Verlinkung innerhalb des PDF-Dokuments, für PDF-Lesezeichen und
% PDF-Metadaten; dieses Paket sollte üblicherweise immer als letztes geladen
% werden
\usepackage[colorlinks=true,allcolors=haw,hyperfootnotes=false,pageanchor=true,linktoc=all]{hyperref}

% für die Druckversion können Sie die obige Zeile durch die folgende ersetzen,
% damit Links nicht blau dargestellt werden:
% \usepackage[draft]{hyperref}

% Metadaten des PDF-Dokumentes; setzen Sie hier Ihren eigenen Namen sowie den
% Titel Ihrer Arbeit ein
\hypersetup{pdfauthor={Darius Daub}}
\hypersetup{pdftitle={Leistungssteuerung mit PWM}}




\newcommand{\norm}[1]{\left\lVert#1\right\rVert}

\DeclareMathAccent{\ring}%
    {\mathalpha}{operators}{"17}
\providecommand*{\angs}%
    {\ensuremath{\smash{\mathrm{\ring A}}}}
\providecommand*{\ohm}%
    {\ensuremath{\mathrm{\Omega}}}
\providecommand*{\degree}%
    {\ensuremath{^\circ}}
\providecommand*{\celsius}%
    {\ensuremath{\mathrm{^\circ C}}}
\providecommand*{\micro}%
    {\ensuremath{\mu}}
\providecommand*{\unit}[1]{%
    \ensuremath{\mathrm{\,#1}}}
% The number `e'
\providecommand*{\eu}%
    {\ensuremath{\mathrm{e}}}
% The imaginary unit
\providecommand*{\iu}%
    {\ensuremath{\mathrm{j}}}
\providecommand*{\ped}[1]{%
    \ensuremath{_\mathrm{#1}}}
\providecommand*{\ap}[1]{%
    \ensuremath{^\mathrm{#1}}}
\providecommand{\newoperator}[3]{%
    \newcommand*{#1}{\mathop{#2}#3}}
\providecommand{\renewoperator}[3]{%
    \renewcommand*{#1}{\mathop{#2}#3}}
\newoperator{\ent}%
    {\mathrm{ent}}{\nolimits}
\renewoperator{\Re}%
    {\mathrm{Re}}{\nolimits}
\renewoperator{\Im}%
    {\mathrm{Im}}{\nolimits}
\makeatletter
\providecommand*{\diff}%
    {\@ifnextchar^{\DIfF}{\DIfF^{}}}
\def\DIfF^#1{%
    \mathop{\mathrm{\mathstrut d}}%
    \nolimits^{#1}\gobblespace}
    \def\gobblespace{%
    \futurelet\diffarg\opspace}
\def\opspace{%
    \let\DiffSpace\!%
    \ifx\diffarg(%
    \let\DiffSpace\relax
    \else
    \ifx\diffarg[%
    \let\DiffSpace\relax
    \else
    \ifx\diffarg\{%
    \let\DiffSpace\relax
    \fi\fi\fi\DiffSpace}
\providecommand*{\deriv}[3][]{%
    \frac{\diff^{#1}#2}{\diff #3^{#1}}}
\providecommand*{\pderiv}[3][]{%
    \frac{\partial^{#1}#2}%
    {\partial #3^{#1}}}